\documentclass{article}

% We suggest
% if you need to pass options to natbib, use, e.g.:
%     \PassOptionsToPackage{numbers, compress}{natbib}
% before loading neurips_2020

% ready for submission
% \usepackage{neurips_2020}

% to compile a preprint version, e.g., for submission to arXiv, add add the
% [preprint] option:
%     \usepackage[preprint]{neurips_2020_tda}

% to compile a camera-ready version, add the [final] option, e.g.:
%     \usepackage[final]{neurips_2020_tda}

% to avoid loading the natbib package, add option nonatbib:
\usepackage[nonatbib]{neurips_2020_tda}

\usepackage[utf8]{inputenc} % allow utf-8 input
\usepackage[T1]{fontenc}    % use 8-bit T1 fonts
\usepackage{amsfonts}       % blackboard math symbols
\usepackage{booktabs}       % professional-quality tables
\usepackage{hyperref}       % hyperlinks
\usepackage{url}            % simple URL typesetting
\usepackage{microtype}      % microtypography
\usepackage{nicefrac}       % compact symbols for 1/2, etc.
\usepackage{paralist}       % in-paragraph enumerations
\usepackage{siunitx}        % SI units

\urlstyle{same}

\title{
  Formatting Instructions for the ``Topological Data Analysis and Beyond''
  Workshop at NeurIPS 2020
}

% The \author macro works with any number of authors. There are two commands
% used to separate the names and addresses of multiple authors: \And and \AND.
%
% Using \And between authors leaves it to LaTeX to determine where to break the
% lines. Using \AND forces a line break at that point. So, if LaTeX puts 3 of 4
% authors names on the first line, and the last on the second line, try using
% \AND instead of \And before the third author name.

\author{%
  David S.~Hippocampus\thanks{Use footnote for providing further information
    about author (webpage, alternative address)---\emph{not} for acknowledging
    funding agencies.} \\
  Department of Computer Science\\
  Cranberry-Lemon University\\
  Pittsburgh, PA 15213 \\
  \texttt{hippo@cs.cranberry-lemon.edu} \\
  % examples of more authors
  % \And
  % Coauthor \\
  % Affiliation \\
  % Address \\
  % \texttt{email} \\
  % \AND
  % Coauthor \\
  % Affiliation \\
  % Address \\
  % \texttt{email} \\
  % \And
  % Coauthor \\
  % Affiliation \\
  % Address \\
  % \texttt{email} \\
  % \And
  % Coauthor \\
  % Affiliation \\
  % Address \\
  % \texttt{email} \\
}

\begin{document}

\maketitle

\begin{abstract}
  This document provides a basic template for your paper and explains
  the submission guidelines. Try to limit your abstract to a single paragraph,
  ideally between 4--6 sentences long.
\end{abstract}

\section{Submission details}

We require electronic submissions via OpenReview. Use the following site
for your submission:
%
\begin{center}
  \url{https://openreview.net/group?id=NeurIPS.cc/2020/Workshop/TDA_and_Beyond}
\end{center}
%
Please read the instructions below carefully and follow them faithfully.
If you have any questions, reach out to \href{mailto:workshop@topology.rocks}{workshop@topology.rocks}.

\section{Style file}

Papers to be submitted to the `Topological Data Analysis and Beyond
workshop' must be prepared according to the instructions presented here.
Papers may only be up to four pages long, including figures.
%
Additional pages containing
%
\begin{inparaenum}[(i)]
  \item \emph{acknowledgements},
  \item \emph{funding statements},
  \item \emph{cited references}, and/or
  \item \emph{supplemental information}
\end{inparaenum}
%
are allowed.
%
Papers that exceed four pages of content~(not counting any appendices or
supplements) will not be reviewed, or in any other way considered for
presentation at the workshop. The margins follow the NeurIPS~2020
template.

This \LaTeX{} style file contains three optional arguments: \verb+final+, which
creates a camera-ready copy, \verb+preprint+, which creates a preprint for
submission to, e.g., arXiv, and \verb+nonatbib+, which will not load the
\verb+natbib+ package for you in case of package clash.

\paragraph{Preprint option}
%
If you wish to post a preprint of your work online, e.g.\ on arXiv, using the
NeurIPS style, please use the \verb+preprint+ option. This will create a
non-anonymous version of your work with the text ``Preprint. Work in progress.''
in the footer. This version may be distributed as you see fit. Please \textbf{do
  not} use the \verb+final+ option, which should \textbf{only} be used for
papers accepted to the workshop.

At submission time, please omit the \verb+final+ and \verb+preprint+
options. This will anonymise your submission and add line numbers to aid
review. Please do \emph{not} refer to these line numbers in your paper as they
will be removed during generation of camera-ready copies.

The file \verb+neurips_2020_tda.tex+ may be used as a ``shell'' for writing your
paper. All you have to do is replace the author, title, abstract, and text of
the paper with your own.
%
The formatting instructions contained in these style files are summarized in
Sections \ref{sec:General formatting instructions}, \ref{sec:Headings},
and \ref{sec:Others} below.

\section{General formatting instructions}
\label{sec:General formatting instructions}

For the final version, authors' names are set in boldface, and each name is
centred above the corresponding address. The lead author's name is to be listed
first~(left-most), and the co-authors' names~(if different address) are set to
follow. If there is only one co-author, list both author and co-author side by
side. Feel free to use footnote marks such as $\dagger$ in order to indicate
equal contributions.

Please pay special attention to the instructions in Section~\ref{sec:Others}
regarding figures, tables, acknowledgements, and references.

\section{Headings}
\label{sec:Headings}

All headings should be lower case (except for first word and proper nouns),
flush left, and bold.

\section{Citations, figures, tables, and references}
\label{sec:Others}

These instructions apply to everyone.

\subsection{Citations within the text}

The \verb+natbib+ package will be loaded for you by default.  Citations may be
author/year or numeric, as long as you maintain internal consistency.  As to the
format of the references themselves, any style is acceptable as long as it is
used consistently.

The documentation for \verb+natbib+ may be found at
\begin{center}
  \url{http://mirrors.ctan.org/macros/latex/contrib/natbib/natnotes.pdf}
\end{center}
Of note is the command \verb+\citet+, which produces citations appropriate for
use in inline text.  For example,
\begin{verbatim}
   \citet{hasselmo} investigated\dots
\end{verbatim}
produces
\begin{quote}
  Hasselmo, et al.\ (1995) investigated\dots
\end{quote}
%
If you wish to load the \verb+natbib+ package with options, you may add the
following before loading the \verb+neurips_2020_tda+ package:
\begin{verbatim}
   \PassOptionsToPackage{options}{natbib}
\end{verbatim}

If \verb+natbib+ clashes with another package you load, you can add the optional
argument \verb+nonatbib+ when loading the style file:
\begin{verbatim}
   \usepackage[nonatbib]{neurips_2020_tda}
\end{verbatim}

As submission is double-blind, you should refer to your own published
work in the third person. That is, use ``In the previous work of Jones
et al.\ [4],'' not ``In our previous work [4].'' If you cite your other
papers that are not widely available (e.g.\ a journal paper under
review), use anonymous author names in the citation, e.g.\ an author of
the form ``A.\ Anonymous.''

\subsection{Footnotes}

Footnotes should be used sparingly.  If you do require a footnote, indicate
footnotes with a number\footnote{Sample of the first footnote.} in the
text.
%
Note that footnotes are properly typeset \emph{after} punctuation
marks.\footnote{As in this example.}

\subsection{Figures}

All artwork must be neat, clean, and legible. Lines should be dark enough for
purposes of reproduction. The figure number and caption always appear after the
figure. Place one line space before the figure caption and one line space after
the figure. The figure caption should be lower case (except for first word and
proper nouns); figures are numbered consecutively.
%
You may use colour figures.  However, it is best for the figure captions
and the paper body to be legible regardless of whether the paper is
printed in black/white or in colour.
%
Moreover, when designing figures, please ensure that your colours can be
distinguished by the colour-blind. Visit
%
\begin{center}
  \url{https://colorbrewer2.org}
\end{center}
%
for more advice on choosing colours.

\subsection{Tables}

All tables must be centred, neat, clean and legible.  The table number and
title always appear before the table.  See Table~\ref{sample-table}.
The table title must be lower case (except for first word and proper
nouns).
%
Note that publication-quality tables \emph{do not contain vertical rules.} We
strongly suggest the use of the \verb+booktabs+ package, which allows for
typesetting high-quality, professional tables:
\begin{center}
  \url{https://www.ctan.org/pkg/booktabs}
\end{center}
This package was used to typeset Table~\ref{sample-table}.

\begin{table}
  \caption{%
  Sample table title
  }
  \label{sample-table}
  \centering
  \begin{tabular}{lll}
    \toprule
    \multicolumn{2}{c}{Part}                   \\
    \cmidrule(r){1-2}
    Name     & Description     & Size (\si{\micro\meter}) \\
    \midrule
    Dendrite & Input terminal  & $\sim$100     \\
    Axon     & Output terminal & $\sim$10      \\
    Soma     & Cell body       & up to $10^6$  \\
    \bottomrule
  \end{tabular}
\end{table}

\section{Final instructions}

Do \textbf{not} change any aspects of the formatting parameters in the style files.  In
particular, do not modify the width or length of the rectangle the text should
fit into, and do not change font sizes~(except perhaps in the
\textbf{References} section; see below). Please note that pages should be
numbered.

\paragraph{Margins in \LaTeX{}}

Most of the margin problems come from figures positioned by hand using
\verb+\special+ or other commands. We suggest using the command
\verb+\includegraphics+ from the \verb+graphicx+ package. Always specify the
figure width as a multiple of the line width as in the example below:
\begin{verbatim}
   \usepackage[pdftex]{graphicx} ...
   \includegraphics[width=0.8\linewidth]{myfile.pdf}
\end{verbatim}
See Section 4.4 in the graphics bundle documentation
(\url{http://mirrors.ctan.org/macros/latex/required/graphics/grfguide.pdf})

\paragraph{Hyphenation}
%
A number of width problems arise when \LaTeX{} cannot properly hyphenate a
line. Please give LaTeX hyphenation hints using the \verb+\-+ command when
necessary.

\section*{Broader Impact}

Authors may optionally include a statement of the broader impact of
their work, including its ethical aspects and future societal
consequences.  Authors may discuss both positive and negative
outcomes, if any. For instance, authors might discuss
%
\begin{inparaenum}[(i)]
  \item who may benefit from this research,
  \item who may be put at disadvantage from this research,
  \item what are the consequences of failure of the system, and
  \item whether the task/method leverages biases in the data.
\end{inparaenum}
%
Use unnumbered first level headings for this section, which should go at
the end of the paper. \textbf{Note that this section does not count towards
the four pages of content that are allowed.}

\begin{ack}
Use unnumbered first level headings for the acknowledgements. All
acknowledgements go at the end of the paper before the list of
references. Moreover, you are required to declare funding~(financial
activities supporting the submitted work) and competing interests
(related financial activities outside the submitted work).
%
More information about this disclosure can be found at:
\url{https://neurips.cc/Conferences/2020/PaperInformation/FundingDisclosure}.

Do \textbf{not} include this section in the anonymised submission, only
in the final paper. You can use the \texttt{ack} environment provided in
the style file to automatically hide this section in an anonymous
submission.
\end{ack}

\section*{References}

References follow the acknowledgements. Use an unnumbered first-level heading for
the references. Any choice of citation style is acceptable as long as you are
consistent. It is permissible to reduce the font size to \verb+small+~(9 point)
when listing the references.
%
\textbf{Note that the Reference section does not count towards the four pages of content that are allowed.}

\end{document}
